\section{Einleitung}\raggedbottom
\label{sec:Einleutung}
In dem ersten Kapitel der Arbeit wollen wir die Problematik und die Aufgabenstellung definieren. Dazu wird zunächst die Firma Computer-Communications Networks (CoCoNet) vorgestellt. Nachdem die Arbeit dadurch in den Firmenkontext gebracht wurde, wollen wir konkret die Fragestellung der Arbeit definieren und deren Aufbau der beschreiben.
\subsection{Firmenprofil CoCoNet GmbH}
\label{sub:Firmenprofil CoCoNet GmbH}
Die Firma CoCoNet entwickelt digitale Banking-Lösungen für Firmenkunden. In dieser Arbeit beschäftigen wir uns mit einem bestimmten CoCoNet Produkt, das Multiversa International Finance Portal (IFP). Das IFP ist eine Server-Basierte Anwendung, auf die User per Internet-Browser zugreifen können. Nachdem sich ein User in das IFP eingeloggt hat, wird er zunächst zum Dashboard weitergeleitet. Das Dashboard ist in gewisser Weise die Startseite des IFPs, bestehend aus einer Navigationsleiste und Dashboardwidgets. Von hier aus haben User die Möglichkeit durch das IFP zu navigieren, um z.B. eine Überweisung zu tätigen. \citep{CoCo20} \\
Für den weiteren Verlauf der Arbeit wollen wir nicht den kompletten Funktionsumfang des IFPs beleuchten, sondern uns auf das Dashboard mit seinen Widgets fokussieren.\\
Ein Widget ist eine Einheit, die auf dem Dashboard abgelegt werden kann und eine bestimmte Funktionalität hat. Betrachten wir als Beispiel das \textit{Open Payments Widget}. Wie der Name schon vermuten lässt, zeigt das Widget Zahlungen an, die noch offen sind, also authorisiert werden müssen. Klickt man nun eine offene Zahlung an, wird man auf die entsprechende Seite weitergeleitet, die mit den Zahlungsinformationen der offenen Zahlung ausgefüllt ist.

\subsection{Problemstellung (Aufgabestellung?) (anforderungen an das system)}
\label{sub:Problemstellung}
In dieser Arbeit geht es darum, das Userverhalten in Hinblick auf die Nutzung der Dash\-boardwidgets im IFP zu analysieren. Da der Begriff Userverhalten noch recht breit gefächert ist, sollen konkret die folgenden Fragen beantwortet werden:
\begin{enumerate}
	\item Wie oft werden Widgets in einem bestimmten Zeitraum benutzt? \label{intro:q1}\\
	\item Lässt sich ein bestimmter Workflow durch die Nutzung der Widgets erkennen?\\
\end{enumerate}
Diese Fragen wollen wir durch die Analyse von Logfiles beantworten.\\
Zunächst müssen wir aber definieren, was es bedeutet, ein Widgets zu benutzen. Da das Dashboard bzw. die Widgets das Erste sind, was man nach einem Login in das IFP sieht, könnte man meinen, dass das pure Vorhandensein auf dem Dashboard eine Nutzung schon einschließt. Betrachtet man z.B. das Widget in [ABB-SCREENSHOT-WIDGET] stellt man fest, dass das Widget schon einiges an Informationen liefert. Allerdings kann man an dieser Stelle nur spekulieren, ob und in welchem Ausmaß der User diese Information wahrnimmt. Da also ein pures Vorhandensein des Widgets auf dem Dashboard als Nutzung nicht hinreichend ist, zählt erst ein Klick auf das Widget als Nutzung.\\
Wenn nun ein Widget angeklickt wird, wird der Nutzer in der Regel auf eine neue Seite weitergeleitet, die durch das Widget z.B. schon vorgefiltert ist. Da diese Weiterleitung ein HTTP-Request an den Server ist, wird sie geloggt und kann in den Logfiles erkannt werden.\\
Also sind die Anforderungen an das zu entwickelnde System, dass die Nutzung eines Widgets erkannt wird und aus diesen Informationen Verhaltensmuster der User bzgl. der eben erwähnten Fragen erkannt werden.\\
Ein weiterer zu berücksichtigender Aspekt ist, dass das System nutzerfreundlich sein soll, damit es auch Menschen ohne besondere Vorkenntnisse benutzen können.

\subsection{Aufbau der Arbeit}
\label{sub:Aufbau der Arbeit}
Im folgenden Verlauf der Arbeit wird zunächst auf die theoretischen Grundlagen der Mustererkennung eingegangen. Dazu wird die Syntax der vorliegenden Logfile Einträge und anschließend die Phasen der Mustererkennung, die die Einträge durchlaufen, beschrieben. In dem darauf folgenden Kapitel wird beschrieben, wie die theoretischen Grundlagen technisch umgesetzt werden. Danach wird das entwickelte System getestet und die Ergebnisse ausgewertet. Zum Schluss wird die Arbeit durch ein Fazit zusammengefasst und zukünftige Einsatzmöglichkeiten erörtert.

\newpage



\begin{comment}

\section{De Bello Gallico}\raggedbottom 

\subsection{Gallia est omnis divisa}
incolunt Belgae, aliam Aquitani, tertiam qui ipsorum lingua
Celtae, nostra Galli appellantur. Hi omnes lingua, institutis,
legibus inter se differunt. Gallos ab Aquitanis Garumna flumen, a
Belgis Matrona et Sequana dividit. Horum omnium fortissimi sunt
Belgae, propterea quod a cultu atque humanitate provinciae
longissime absunt, minimeque ad eos mercatores saepe commeant
atque ea quae ad effeminandos animos pertinent important,
proximique sunt Germanis, qui trans Rhenum incolunt, quibuscum
continenter bellum gerunt. Qua de causa Helvetii quoque reliquos
Gallos virtute praecedunt, quod fere cotidianis proeliis cum
Germanis contendunt, cum aut suis finibus eos prohibent aut ipsi
in eorum finibus bellum gerunt. Eorum una, pars, quam Gallos
obtinere dictum est, initium capit a flumine Rhodano, continetur
Garumna flumine, Oceano, finibus Belgarum, attingit etiam ab
Sequanis et Helvetiis flumen Rhenum, vergit ad septentriones.
Belgae ab extremis Galliae finibus oriuntur, pertinent ad
inferiorem partem fluminis Rheni, spectant in septentrionem et
orientem solem. Aquitania a Garumna flumine ad Pyrenaeos montes et
eam partem Oceani quae est ad Hispaniam pertinet; spectat inter
occasum solis et septentriones.


\subsection{Apud Helvetios longe} Apud Helvetios longe nobilissimus fuit et
ditissimus
Orgetorix. Is M. Messala, [et P.] M. Pisone consulibus regni
cupiditate inductus coniurationem nobilitatis fecit et civitati
persuasit ut de finibus suis cum omnibus copiis exirent: perfacile
esse, cum virtute omnibus praestarent, totius Galliae imperio
potiri. Id hoc facilius iis persuasit, quod undique loci natura
Helvetii continentur: una ex parte flumine Rheno latissimo atque
altissimo, qui agrum Helvetium a Germanis dividit; altera ex parte
monte Iura altissimo, qui est inter Sequanos et Helvetios; tertia
lacu Lemanno et flumine Rhodano, qui provinciam nostram ab
Helvetiis dividit. His rebus fiebat ut et minus late vagarentur et
minus facile finitimis bellum inferre possent; qua ex parte
homines bellandi cupidi magno dolore adficiebantur. Pro
multitudine autem hominum et pro gloria belli atque fortitudinis
angustos se fines habere arbitrabantur, qui in longitudinem milia
passuum CCXL, in latitudinem CLXXX patebant.

His rebus adducti et auctoritate Orgetorigis permoti constituerunt
ea quae ad proficiscendum pertinerent comparare, iumentorum et
carrorum quam maximum numerum coemere, sementes quam maximas
facere, ut in itinere copia frumenti suppeteret, cum proximis
civitatibus pacem et amicitiam confirmare. Ad eas res conficiendas
biennium sibi satis esse duxerunt; in tertium annum profectionem
lege confirmant. Ad eas res conficiendas Orgetorix deligitur. Is
sibi legationem ad civitates suscipit. In eo itinere persuadet
Castico, Catamantaloedis filio, Sequano, cuius pater regnum in
Sequanis multos annos obtinuerat et a senatu populi Romani amicus
appellatus erat, ut regnum in civitate sua occuparet, quod pater
ante habuerit; itemque Dumnorigi Haeduo, fratri Diviciaci, qui eo
tempore principatum in civitate obtinebat ac maxime plebi acceptus
erat, ut idem conaretur persuadet eique filiam suam in matrimonium
dat. Perfacile factu esse illis probat conata perficere, propterea
quod ipse suae civitatis imperium obtenturus esset: non esse
dubium quin totius Galliae plurimum Helvetii possent; se suis
copiis suoque exercitu illis regna conciliaturum confirmat. Hac
oratione adducti inter se fidem et ius iurandum dant et regno
occupato per tres potentissimos ac firmissimos populos totius
Galliae sese potiri posse sperant.

\ifthenelse{\boolean{\biber}}{ % Beispiel um mit Biber zu zitieren (\citet und \citep)
	\citet{Con97} hat ein Buch geschrieben. Es gibt auch andere Arbeiten \citep{PeHe97} die referenziert sind. In Abbildung \ref{fig_Gallien} ist ein Sachverhalt dargestellt.
	
	
	1 Autor: \citet{Con97} \hspace*{1cm} \citep{Con97}\\
	2 Autoren: \citet{IWNLP} \hspace*{1cm} \citep{IWNLP}\\
	3 Autoren: \citet{liebeck-esau-conrad:2016:ArgMining2016} \hspace*{1cm} \citep{liebeck-esau-conrad:2016:ArgMining2016}
	
	Online resource: \citet{ILSVRC2016}
}{ %  Beispiel um klassisch zu zitieren (\cite)
	\cite{Con97} hat ein Buch geschrieben. Es gibt auch andere Arbeiten \cite{PeHe97} die referenziert sind. In Abbildung \ref{fig_Gallien} ist ein Sachverhalt dargestellt.
	
	
	1 Autor: \cite{Con97} \hspace*{1cm} \cite{Con97}\\
	2 Autoren: \cite{IWNLP} \hspace*{1cm} \cite{IWNLP}\\
	3 Autoren: \cite{liebeck-esau-conrad:2016:ArgMining2016} \hspace*{1cm} \cite{liebeck-esau-conrad:2016:ArgMining2016}
	
	Online resource: \cite{ILSVRC2016}
}

\textbf{quotes}:\\
\glqq quote\grqq.

\begin{figure}[htb]
\begin{center}
  \includegraphics[width=175pt, angle=270]{bilder/Galia}
  \caption{Gallien zur Zeit Caesars}\label{fig_Gallien}
\end{center}
\end{figure}


\begin{table}[htb]
\begin{center}
\begin{tabular}{|l|l|l|}
\hline
Jahr &  Erster Consul & Zweiter Consul\\
\hline \hline
1 & C. Caesar         & L. Aemilius Paullus\\
2 & P. Vinicius       & P. Alfenus Varus\\
3 & L. Aelius Lamia   & M. Servilius\\
4 & Sex. Aelius Catus &  C. Sentius Saturninus\\
5 & L. Valerius Messalla& Cn. Cornelius Cinna \\
suff. & C. Vibius Postumus &  C. Ateius Capito\\
6 & M. Aemilius Lepidus & L. Arruntius\\
\hline
\end{tabular}
 \caption{Römische Konsulen}\label{tab_Konsulen}
\end{center}
\end{table}


\pagebreak

\section{De Bello Hispaniensi}\raggedbottom 

Pharnace superato, Africa recepta, qui ex his proeliis cum
adulescente Cn. Pompeio profugissent, cum . . . et ulterioris
Hispaniae potitus esset, dum Caesar muneribus dandis in Italia
detinetur, . . . quo facilius praesidia contra compararet,
Pompeius in fidem uniuscuiusque civitatis confugere coepit. Ita
partim precibus partim vi bene magna comparata manu provinciam
vastare. Quibus in rebus non nullae civitates sua sponte auxilia
mittebant, item non nullae portas contra cludebant. Ex quibus si
qua oppida vi ceperat, cum aliquis ex ea civitate optime de Cn.
Pompeio meritus civis esset, propter pecuniae magnitudinem alia
qua ei inferebatur causa, ut eo de medio sublato ex eius pecunia
latronum largitio fieret. Ita pacis commoda hoste +hortato+
maiores augebantur copiae. +Hoc crebris nuntiis in Italiam missis
civitates contrariae Pompeio+ auxilia sibi depostulabant.

C. Caesar dictator tertio, designatus dictator quarto multis
+iterante diebus coniectis+ cum celeri festinatione ad bellum
conficiendum in Hispaniam cum venisset, legatique Cordubenses, qui
a Cn. Pompelo discessissent, Caesari obviam venissent, a quibus
nuntiabatur nocturno tempore oppidum Cordubam capi posse, quod nec
opinantibus adversariis eius provinciae potitus esset, simulque
quod tabellariis, qui a Cn. Pompeio dispositi omnibus locis
essent, qui certiorem Cn. Pompeium de Caesaris adventu facerent .
. . multa praeterea veri similia proponebant. Quibus rebus
adductus quos legatos ante exercitui praefecerat Q. Pedium et Q.
Fabium Maximum de suo adventu facit certiores, utque sibi
equitatus qui ex provincia fuisset praesidio esset. Ad quos
celerius quam ipsi opinati sunt appropinquavit neque, ut ipse
voluit, equitatum sibi praesidio habuit.

Erat idem temporis Sex. Pompeius frater qui cum praesidio Cordubam
tenebat, quod eius provinciae caput esse existimabatur; ipse autem
Cn. Pompeius adulescens Uliam oppidum oppugnabat et fere iam
aliquot mensibus ibi detinebatur. Quo ex oppido cognito Caesaris
adventu legati clam praesidia Cn. Pompei Caesarem cum adissent,
petere coeperunt uti sibi primo quoque tempore subsidium mitteret.
Caesar - eam civitatem omni tempore optime de populo Romano
meritam esse - celeriter sex cohortis secunda vigilia iubet
proficisci, pari equites numero. Quibus praefecit hominem eius
provinciae notum et non parum scientem, L. Vibiurn Paciaecum. Qui
cum ad Cn. P praesidia venisset, incidit idem temporis ut
tempestate adversa vehementique vento adflictaretur; aditusque vis
tempestatis ita obscurabat ut vix proximum agnoscere possent.
Cuius incommodum summam utilitatem ipsis praebebat. Ita cum ad eum
locum venerunt, iubet binos equites conscendere, et recta per
adversariorum praesidia ad oppidum contendunt. Mediisque eorum
praesidiis cum essent, cum quaereretur qui essent unus ex nostris
respondit, ut sileat verbum facere: nam id temporis conari ad
murum accedere, ut oppidum capiant; et partim tempestate impediti
vigiles non poterant diligentiam praestare, partim illo responso
deterrebantur. Cum ad portam appropinquassent, signo dato ab
oppidanis sunt reccepti, et pedites dispositi partim ibi
remanserunt, equites clamore facto eruptionem in adversariorum
castra fecerunt.

\pagebreak
\section{Weiteres Kapitel}\raggedbottom 
\subsection{Unterkapitel}
\subsection{Unterkapitel}
\end{comment}
